% Part 1: Determining RecB binding time
\section*{Results}

\subsection*{RecB forms long-lived spots when recruited to cipro\-floxacin DSBs}

To quantify RecB binding time to DNA in live \ecoli, we used a Halo-tag fusion to the RecB subunit, conjugated to the JF549 fluorescent dye (Figure \ref{Fig:exp_principle}A). The fusion was previously used and characterised, ensuring specific one-to-one labelling of RecB molecules without adverse effects on the DNA repair process\cite{Lepore2019a,Lepore2023}. Due to the large size and topological constraints of the bacterial chromosome, DNA-bound RecB diffuses very slowly\cite{Lepore2023}. We therefore applied the previously developed technique of localisation enhancement\cite{Yu2006, Elf2007}: since RecB is present at low copy numbers in \ecoli\ ($\sim$5 molecules per cell on average\cite{Lepore2019a}), imaging live cells with a long exposure time (1 second) made fast-diffusing RecB molecules appear as weak homogeneous signal in the cell, while immobile or slow-diffusing RecB molecules formed intense diffraction-limited spots (referred to throughout this article as RecB spots, Figures \ref{Fig:exp_principle}A and \ref{Fig:exp_principle}B). The complete absence of similar spots in cells expressing the free Halo-tag from a plasmid confirmed that these spots were specific to RecB (Supp. Figure \ref{SIFig:freehalo_image}). However, their exact nature needed to be confirmed. Although DNA-bound molecules are expected to diffuse very slowly, the RecBCD-Halo complex diffuses in the crowded and constrained environment of the cytoplasm, and might undergo transient interactions with DNA\cite{Lepore2023}, which could lead to RecB spots forming independently of DNA binding. To gain more insight into the nature of RecB spots, we increased the number of induced DSBs by adding ciprofloxacin to agar pads before imaging, and measured the lifetime of the spots by recording 100-second timelapse videos with 2 sec interval between frames on different positions in the sample, for a total time of 60 min following exposure to ciprofloxacin (Figure \ref{Fig:lifetimes}A). The 1 second exposure time was optimal to detect RecB spots with good signal-to-noise, while limiting dye photobleaching and keeping a sufficient time resolution to determine spot lifetimes. Visual inspection of the timelapse data showed that RecB spot lifetime was hetereogenous, with a majority of spots being visible only for a few frames, and a small number persisting for longer times. These longer-lived spots became more frequent as the ciprofloxacin concentration was increased. Inspecting background-subtracted intensity time-traces for single RecB spots showed that spots lost their intensity in a single step (Supp. Figure \ref{SIFig:SM_traces}), consistent with single-molecules photobleaching or transitioning from a very slowly diffusing state to a rapid one. Our imaging setup therefore allowed us to visualise single very slowly diffusing RecB molecules, and to measure the time they spent in this diffusive state.

\begin{figure}[htbp]
    \centering
    \includegraphics[width=.45\textwidth]{Figures/Fig1_Exp_principle.pdf}
    \caption{Detecting RecB binding to DSBs. \textbf{(A)} Scheme of our experimental protocol. The RecB subunit of the RecBCD complex is fused to a Halo-tag, bound by the JF549 fluorescent dye\cite{Lepore2019a, Lepore2023}. A long exposure time (1 sec) makes diffusing molecules appear as diffuse signal in the cell, while DNA-bound molecules are visible as bright, diffraction-limited spots. \textbf{(B)} Example image of a RecB spot (white arrow).}
    \label{Fig:exp_principle}
\end{figure}

\subsection*{RecB dissociation time is not affected by the concentration of cipro\-floxacin}

% RecB spot lifetime
Using the timelapse data, we built histograms of RecB spot lifetimes at the different ciprofloxacin concentrations (Figure \ref{Fig:lifetimes}B). In the absence of ciprofloxacin, a large majority of spots lasted less than 10 seconds. Under increasing ciprofloxacin concentrations, the distribution showed a clear shift towards longer-lived spots, with the histogram at 20 and 30 ng/ml ciprofloxacin displaying a clear "tail" of long-lived events ($>$ 10 sec). Fitting these histograms with a mono-exponential decay model ($y = a.e^{-k.t}$) did not match the experimental data accurately (Supp. Figure \ref{SIFig:monoexp_fits}). In particular, it did not account for the tail of longer-lived spots that formed at high ciprofloxacin concentrations. Fitting with a bi-exponential decay model ($y = a_1.e^{-k_1.t} + a_2.e^{-k_2.t}$) accounted better for the longer-lived RecB spots (Figure \ref{Fig:lifetimes}B). The bi-exponential fit outlined two populations of spots with different lifetimes: a short-lived one, with average lifetimes ranging from 1.4 to 1.7 sec, and a longer-lived one, with average lifetimes ranging from 10 to 14 sec (Table \ref{tab:fit_results}). Even though the short-lived spots always represented a majority of events (over 90\% of the spots), the proportion of long-lived spots tended to increase under higher ciprofloxacin exposure (from 1.4 $\pm$ 0.3\% at 3 ng/ml ciprofloxacin to 5.5 $\pm$ 1.4\% at 30 ng/ml ciprofloxacin).

\begin{figure*}[htbp]
    \centering
    \includegraphics[width=.8\textwidth]{Figures/Fig2_RecB_lifetime.pdf}
    \caption{RecB lifetime on DNA under ciprofloxacin exposure. \textbf{(A)} Example kymographs (2 sec interval, 100 sec total) of cells containing short- and long-lived RecB spots. \textbf{(B)} RecB spot lifetime histograms at 0, 3, 10, 20 and 30 ng/ml ciprofloxacin (bars), fitted with a bi-exponential decay model (black line, fit components showed as dashed lines). \ncells{66,764}. \nspots{170,138}. \textbf{(C)} RecB spot lifetime histograms (bars) for cells over-expressing the Gam protein, fitted with a bi-exponential decay model (black line, fit components showed as dashed lines). \ncells{8,812}. \nspots{18,698}. \textbf{(D)} Probability of a spot corresponding to a DNA-bound RecB molecule, for cells over-expressing the Gam protein. Black dots show averages for individual datasets, the black line is the average between them, and the red dashed line shows the smallest lifetime at which RecB spots have 95\% probability to be DNA-bound. \ncells{8,812}. \nspots{18,698}.}
    \label{Fig:lifetimes}
\end{figure*}

\begin{table}[htbp]
    \centering
    \caption{Parameters derived from the spot lifetime histogram fits (Figures \ref{Fig:lifetimes}B and \ref{Fig:lifetimes}C). The lifetime was calculated as the inverse of the fitted dissociation rate. Values are given as the median $\pm$ standard deviation over at least 3 independent datasets. \ncells{66,764}. \nspots{170,138}}
    \begin{tabular}{lllll}
        \toprule
        &  &  & \makecell{Lifetime\\(sec)} & \makecell{Population\\(\%)} \\
        Strain & \makecell{Cipro.} & Type &  &  \\
        \midrule
        \multirow[t]{10}{*}{WT} & \multirow[t]{2}{*}{0 ng/ml} & Short & 1.4 $\pm$ 0.2 & 98.2 $\pm$ 1.0 \\
        &  & Long & 9.5 $\pm$ 9.7 & 1.8 $\pm$ 1.0 \\
        \cline{2-5}
        & \multirow[t]{2}{*}{3 ng/ml} & Short & 1.5 $\pm$ 0.1 & 98.6 $\pm$ 0.3 \\
         &  & Long & 10.5 $\pm$ 1.3 & 1.4 $\pm$ 0.3 \\
        \cline{2-5}
        & \multirow[t]{2}{*}{10 ng/ml} & Short & 1.7 $\pm$ 0.2 & 97.6 $\pm$ 0.6 \\
        &  & Long & 13.7 $\pm$ 2.3 & 2.4 $\pm$ 0.6 \\
        \cline{2-5}
        & \multirow[t]{2}{*}{20 ng/ml} & Short & 1.6 $\pm$ 0.2 & 96.6 $\pm$ 0.8 \\
        &  & Long & 11.8 $\pm$ 2.1 & 3.4 $\pm$ 0.8 \\
        \cline{2-5}
        & \multirow[t]{2}{*}{30 ng/ml} & Short & 1.7 $\pm$ 0.2 & 94.5 $\pm$ 1.4 \\
        &  & Long & 13.3 $\pm$ 2.2 & 5.5 $\pm$ 1.4 \\
        \midrule
        \multirow[t]{4}{*}{Gam} & \multirow[t]{2}{*}{0 ng/ml} & Short & 1.7 $\pm$ 0.2 & 99.2 $\pm$ 0.4 \\
        &  & Long & 15.5 $\pm$ 9.1 & 0.8 $\pm$ 0.4 \\
        \cline{2-5}
        & \multirow[t]{2}{*}{30 ng/ml} & Short & 1.5 $\pm$ 0.1 & 98.2 $\pm$ 0.9 \\
        & & Long & 8.2 $\pm$ 1.8 & 1.8 $\pm$ 0.9 \\
        \bottomrule
        \end{tabular}
    \label{tab:fit_results}
\end{table}

% Gam
To determine whether short- or long-lived spots resulted from RecB binding to DNA leading to SOS induction, we measured RecB spot lifetime in the presence and absence of ciprofloxacin while over-expressing the Gam protein of phage $\lambda$ from a plasmid. The Gam protein was previously shown to bind in place of DNA on the RecBCD complex\cite{Wilkinson2016}, and its overexpression is expected to abolish RecBCD binding to DSBs. Accordingly, cells that over-expressed Gam and were exposed to high ciprofloxacin (30 ng/ml) showed little elongation compared to cells that did not overexpress Gam (Supp. Figure \ref{SIFig:Gam_cell_length}), indicating that most cells did not induce the SOS response, due to the inability of the RecBCD-Gam complex to bind to DSBs and load RecA. The resulting RecB spot lifetimes showed a similar distribution to the one obtained in the absence of ciprofloxacin (Supp. Figure \ref{SIFig:Gam_RecB_lifetimes_vs_WT}). This was confirmed by fitting the histogram with our bi-exponential decay model (Figure \ref{Fig:lifetimes}C), which found a proportion of long-lived spots in the presence of 30 ng/ml ciprofloxacin (1.8 $\pm$ 0.9\%, Table \ref{tab:fit_results}) equivalent to wild-type cells that were not exposed to ciprofloxacin (1.8 $\pm$ 1\%). The residual amount of long-lived RecB spots in the presence of Gam could be due to residual DNA binding despite the presence of Gam. This residual binding could also be the cause for the small increase in cell length observed in Gam-expressing cells under 30 ng/ml ciprofloxacin (Supp. Figure \ref{SIFig:Gam_cell_length}). Since Gam overexpression prevents RecB binding to DNA ends, and specifically caused long-lived spots to disappear, we can conclude that long-lived spots correspond to RecB molecules that have bound to a DSB. Given that under Gam overexpression, 95\% of RecB spots were shorter than 10 sec (Figure \ref{Fig:lifetimes}D), we considered any spot with a lifetime over 10 sec to be a DNA-bound RecB molecule. The appearance of short-lived spots in the presence of Gam is likely due to a combination of short-lived DNA binding and slow, confined diffusion of the RecBCD-Gam-Halo complex (Supp. Note \ref{note:spurious_spots} and Supp. Figure \ref{SIFig:displacement_simul}). Therefore, spots that are visible for less than 10 seconds cannot be reliably assigned as either DNA-bound or not. However, the largely reduced cell elongation in the presence of Gam (Supp. Figure \ref{SIFig:Gam_cell_length}) indicates that short-lived spots do not trigger the SOS response.

Using the Gam protein allowed us to identify DNA-bound RecB molecules as the longer-lived population in our bi-exponential fits of the RecB spot lifetime histograms. We determined that when it processes a DSB, RecB stays bound to DNA for 10 to 15 seconds on average, regardless of the ciprofloxacin concentration (Table \ref{tab:fit_results}). This is consistent with processing of DSBs by RecBCD taking place independently of the presence of other DSBs in the cell. In addition, to test whether the duration of ciprofloxacin exposure had an effect on spot lifetimes, we performed separate fits at 15 min intervals (Supp. Figure \ref{SIFig:RecB_lifetimes_timepoints}). The RecB spot lifetimes obtained were consistent with those reported in Table \ref{tab:fit_results} (10-15 sec for long-lived spots, and 1.5-2 sec for short-lived spots), indicating that RecB binding was not affected by the duration of ciprofloxacin exposure.

\subsection*{RecB dissociation time is independent of RecA loading}

% Effect of ciprofloxacin exposure on the different mutants
Next, we wanted to test whether RecB dissociation from DNA was influenced by the following step in the repair process, the loading of RecA on DNA. Therefore, we imaged the \dreca\ and \geneteneighty\ mutants in the presence and absence of 30 ng/ml ciprofloxacin. Supp. Figure \ref{SIFig:mutants_bf} shows defocussed brightfield images of wild-type and mutant cells that were not exposed to ciprofloxacin, or exposed to 30 ng/ml ciprofloxacin for 60 min. Supp. Figure \ref{SIFig:mutants_cell_lengths} shows the average cell lengths for the different strains and ciprofloxacin exposure durations. As previously observed, wild-type cells show a clear response to exposure to 30 ng/ml ciprofloxacin, where the cell length increases significantly due to induction of the SOS response. In contrast to this, since the \dreca\ mutant is unable to form a RecA filament, it is also unable to induce the SOS response. Cell length therefore remains unchanged, even upon prolonged exposure to 30 ng/ml ciprofloxacin. In the absence of ciprofloxacin, the \geneteneighty\ mutant cells were slightly more elongated than wild-type cells. This matches our previous observations that this mutant has a higher baseline of SOS induction than wild-type cells in the absence of exogenous DNA damage\cite{Lepore2023}. Upon exposure to ciprofloxacin, the cell length of the \geneteneighty\ mutant increased, albeit slightly less on average than the wild-type's (Supp. Figure \ref{SIFig:mutants_cell_lengths}). This might be a reflection of \teneighty's inability to directly load RecA on the DNA, and the use of the alternative RecFOR pathway leading to different time dynamics of the SOS response\cite{Ivancic-Bace_2003,Lepore2023}.

% RecB binding time in the mutants
As for the wild-type strain, we imaged RecB, computed histograms of RecB spot lifetimes and fitted them with a bi-exponential decay model (Supp. Figure \ref{SIFig:mutants_biexp_fits}), from which we extracted the binding times of RecB on DNA (Table \ref{tab:fit_mutants}). In the \dreca\ mutant, the DNA-binding lifetime of RecB was similar to the wild-type ($\sim$12 sec), both in the presence and absence of ciprofloxacin. This suggests that RecA loading by RecBCD does not play a role in triggering the dissociation of wild-type RecBCD from DNA. In contrast, \teneighty\ stayed bound to DNA for longer on average than wild-type RecB ($\sim$16 to 19 sec). This could be due to the inability of \teneighty\ to digest the DNA it unwinds, making dissociation more challenging due to the DNA strands threading through the RecBCD complex.

% Table 2: RecB spot lifetime in the mutants
\begin{table}[htbp]
    \centering
    \caption{Parameters derived from the spot lifetime histogram fits for the \dreca\ and \geneteneighty\ strains (Supp. Figure \ref{SIFig:mutants_biexp_fits}). The lifetime was calculated as the inverse of the fitted dissociation rate. Values are given as the median $\pm$ standard deviation over at least 3 independent datasets. \ncells{56,131}. \nspots{177,646}.}
    \begin{tabular}{lllll}
        \toprule
        &  &  & \makecell{Lifetime\\(sec)} & \makecell{Population\\(\%)} \\
        Strain & Cipro. & Type &  &  \\
        \midrule
        \multirow[t]{4}{*}{\dreca} & \multirow[t]{2}{*}{0 ng/ml} & Short & 1.8 $\pm$ 0.2 & 96.6 $\pm$ 1.0 \\
        &  & Long & 13.0 $\pm$ 1.8 & 3.5 $\pm$ 1.0 \\
        \cline{2-5}
        & \multirow[t]{2}{*}{30 ng/ml} & Short & 2.1 $\pm$ 0.2 & 90.5 $\pm$ 3.4 \\
        &  & Long & 12.4 $\pm$ 3.5 & 9.5 $\pm$ 3.4 \\
        \cline{1-5} \cline{2-5}
        \multirow[t]{4}{*}{$recB_{1080}$} & \multirow[t]{2}{*}{0 ng/ml} & Short & 1.8 $\pm$ 0.2 & 97.4 $\pm$ 0.9 \\
        &  & Long & 16.7 $\pm$ 5.5 & 2.6 $\pm$ 0.9 \\
        \cline{2-5}
        & \multirow[t]{2}{*}{30 ng/ml} & Short & 2.1 $\pm$ 0.1 & 96.5 $\pm$ 0.7 \\
        &  & Long & 19.4 $\pm$ 7.1 & 3.5 $\pm$ 0.7 \\
        \bottomrule
    \end{tabular}
    \label{tab:fit_mutants}
\end{table}

% Part 2: Rate of recruitment of RecB on the DNA
\subsection*{RecB binding can be used to estimate the rate of DSB formation}

\begin{figure}[htbp]
    \centering
    \includegraphics[width=.4\textwidth]{Figures/Fig3_RecB_recruitment.pdf}
    \caption{Recruitment of RecB to DSBs. \textbf{(A)} Recruitment rate of RecB on DNA per cell at different ciprofloxacin concentrations. Black points show averages for individual datasets, and bars the median value between them. \ncells{66,764}. \nspots{170,138}. \textbf{(B)} Recruitment rate of RecB on DNA per cell for the \dreca\ and \geneteneighty\ mutants. Black points show averages for individual datasets, and bars the median value between them. \ncells{23,540}. \nspots{99,891}.}
    \label{Fig:recruitment}
\end{figure}

% RecB recruitment in WT
Whereas the lifetime of RecB spots informs us on how long RecB stays bound to DNA, the rate of appearance of spots helps us estimate the rate of recruitment of RecB to DSBs under each DNA damage condition. To do so, we used the RecB spot lifetime histogram fits to estimate the total number of slow-dissociating spots per timelapse. Since we determined that the appearance of a slow-dissociating spot corresponds to the binding of RecB to a DSB, this allowed us to calculate the number of RecB recruitments to DSBs per cell per hour (Figure \ref{Fig:recruitment}A). We expect that the recruitment rate of RecB to DSBs gives a close estimate of the rate of formation of DSBs by ciprofloxacin (see discussion). We estimated that under endogenous damage or low ciprofloxacin concentrations, 1 to 2 RecB molecules are recruited to DSBs per hour on average. A previous study had estimated that 18\% of cells generated an endogenous DSB per cell cycle\cite{Sinha2018}. Considering the division time in our growth conditions (35 min), this would correspond to a rate of $\sim$0.3 DSBs per hour, a value roughly consistent with our own estimate. As expected, the recruitment rate of RecB scaled with ciprofloxacin concentration, up to 8 RecB recruited to DNA per hour at 30 ng/ml. It has been reported that DSB repair from the point of RecBCD binding takes $\sim$15 min in \ecoli\cite{Wiktor2021}; therefore, it appears consistent that undergoing 8 DSBs per hour at 30 ng/ml ciprofloxacin (a concentration higher than the Minimum Inhibitory Concentration, MIC) will cause multiple DSBs to be repaired simultaneously, and be lethal to most cells.

% RecB recruitment in mutants
Estimating the recruitment rate of RecB to DNA provided additional insight into the dynamics of DSB processing in the \dreca\ and \geneteneighty\ mutants (Figure \ref{Fig:recruitment}B). In the \dreca\ mutant, the rate of recruitment of RecB to DSBs was increased compared to wild-type cells, both in the presence and absence of ciprofloxacin. As in both cases the rate of DSB formation is expected to be the same between the wild-type and the \dreca\ mutant, we can hypothesise that this higher rate of RecB recruitment to DNA is due to multiple recruitment events on the same original DSB that cannot be repaired. In the mutant cells, RecBCD processes the DNA, but the generated ssDNA cannot be coated with the RecA protein, and is therefore likely to be degraded by cellular endo- and exonucleases such as SbcCD and ExoI\cite{Zahradka2009}. Degradation of the ssDNA would lead to a blunting of the DNA end, hence creating a new substrate for RecBCD binding. This deleterious cycle was previously reported to occur in cells lacking the RecA protein, eventually leading to full chromosome degradation\cite{Capaldo1975,Skarstad1993}. In our experiment, it results in multiple recruitments of RecB to DNA for each DSB.

% Recruitment in RecB1080
In the \geneteneighty\ mutant, RecB recruitment is higher than in WT cells, but equivalent (in the absence of ciprofloxacin) or lower (in the presence of 30 ng/ml ciprofloxacin) to the level of RecB recruitment in the \dreca\ mutant (Figure \ref{Fig:recruitment}B). Upon DSB recognition, \teneighty\ unwinds DNA without digesting it, and dissociates $\sim$1.5 times slower than wild-type RecB (Tables \ref{tab:fit_results} and \ref{tab:fit_mutants}). After RecBCD dissociation, we expect two competing pathways to take place. Either (i) the unwound ssDNA is digested by cellular endo- and exonucleases, creating a new RecBCD substrate; or (ii) RecFOR displaces the SSB (Single-Strand DNA-Binding) protein and promotes RecA loading, allowing DNA repair by homologous recombination to proceed\cite{Ivancic-Bace_2003}. As a result, \teneighty\ can be recruited on DNA several times for each DSB (similarly to the \dreca\ mutant), but the cycle of re-recruitment can be broken by the action of RecFOR.

% Part 3: Imaging RecA foci and filaments
\subsection*{Ciprofloxacin exposure induces formation of RecA filaments and nucleoid compaction}

During DSB processing, RecBCD facilitates the loading of the RecA protein on single-stranded DNA. To broaden our view of the repair process downstream of RecBCD, we imaged a tandem fusion of RecA with the fluorescent protein SYFP2 which was previously shown to preserve RecA functionality\cite{Wiktor2021}. Based on the fluorescence distribution in the cells (Supp. Figure \ref{SIFig:reca_structures}A), we identified three states of RecA: diffuse, forming a bright focus, or forming an elongated filament, which matches previous \emph{in-vivo} observations of RecA\cite{Wiktor2021}. Because of the large diversity of shapes observed, especially for RecA filaments, the detection of RecA structures by rule-based algorithms was challenging. Therefore, we designed and trained a deep-learning algorithm capable of classifying individual cells based on the three types of spatial distributions cited above (Supp. Figures \ref{SIFig:object_class} and \ref{SIFig:reca_structures}B). In cells that were not exposed to ciprofloxacin, the RecA-associated fluorescence was mostly diffuse, and occasionally formed a bright focus or filament. This is consistent with the expectation that RecA diffuses freely in the cell in the absence of DNA damage, and only polymerises on DNA in the event of an endogenous DSB. After one hour of exposure, the proportion of cells that contained a RecA filament had increased to $\sim$40\% at 20 ng/ml ciprofloxacin, and $\sim$60\% at 30 ng/ml. RecA foci on the other hand formed quickly following ciprofloxacin exposure (present in $\sim$40\% of the cells after 15 min of exposure to 30 ng/ml ciprofloxacin) and came back to endogenous damage level after $\sim$1 hour. Taken together, these results show that RecA foci are transient structures in the repair process that do not accumulate under high DNA damage, whereas RecA filaments accumulate under constant exposure to ciprofloxacin (see Discussion).

% Part 4: Colocalisation of RecB with the nucleoid
\subsection*{DNA-bound RecB colocalises with the bacterial nucleoid}

\begin{figure*}[htbp]
    \centering
    \includegraphics[width=.8\textwidth]{Figures/Fig4_nucleoid.pdf}
    \caption{Colocalisation of RecB spots with the bacterial nucleoid. \textbf{(A)} Representative images of cells (segmented outline in grey) showing the nucleoid (green) and RecB-associated fluorescence (magenta). RecB spots (indicated by arrows) are located in close proximity to the nucleoid. \textbf{(B)} Overlay of nucleoid density (green area) and position of DNA-bound RecB molecules (magenta bars) along the cell's long axis, for different ciprofloxacin concentrations (0 to 30 ng/ml). \ncells{24,014}. \nspots{79,969}. \nnucl{31,441}. \textbf{(C)} Representative overlay images of RecB-associated fluorescence (magenta) and the bacterial nucleoid (green) in the \dreca\ and \geneteneighty\ mutants. Segmented cell outline shown in grey. \textbf{(D)} Overlay of nucleoid density (green area) and position of DNA-bound RecB molecules (magenta bars) along the cell's long axis for the \dreca\ and \geneteneighty\ mutants at 0 and 30 ng/ml ciprofloxacin. \ncells{7,804}. \nspots{36,595}. \nnucl{11,215}.}
    \label{Fig:nucleoid}
\end{figure*}

% Nucleoid imaging
RecA loading on ssDNA triggers the SOS response, which inhibits cell division, leading to cell filamentation. Previous studies have reported that the SOS response also triggers compaction of the bacterial nucleoid\cite{Odsbu2014}. To see if this was the case for breaks induced by ciprofloxacin in our experimental conditions, we stained DNA using the Sytox Green dye. The use of a green dye allowed us to concomitantly image RecB, and to correlate the position of DNA-bound RecB molecules (defined as RecB spots with a lifetime $>$10 sec, see Figure \ref{Fig:lifetimes}D) with that of the nucleoid. Figure \ref{Fig:nucleoid}A shows representative images of the nucleoid and DNA-bound RecB after 60 minutes of exposure to different concentrations of ciprofloxacin. In untreated cells, the nucleoid was often observed to be bi-lobed, consistently with our previous observations\cite{Lepore2023}. After exposure to ciprofloxacin, the cells appeared elongated, and the nucleoid was often compacted in the centre of the cell. Due to nucleoid compaction, the total fraction of the cell occupied by the nucleoid decreased, from 40\% on average in untreated cells to 31\% in cells exposed to 30 ng/ml of ciprofloxacin for an hour (Supp. Figure \ref{SIFig:nucleoid_compaction}). In addition to the compaction, we observed that nucleoids were increasingly centred and the cells elongated upon increasing exposure to ciprofloxacin (in time and concentration), as shown in Supp. Figure \ref{SIFig:nucleoid_position}.

% Correlation of nucleoid and RecB spots
Long-lived RecB spots were found in close proximity to the nucleoid (Figure \ref{Fig:nucleoid}A), which strengthens our interpretation that they are DNA-bound RecB molecules. Figure \ref{Fig:nucleoid}B shows the spatial distribution of DNA-bound RecB spots and the nucleoid density in the cell (Supp. Figure \ref{SIFig:recb_nucleoid_timepoints} shows the same data after different durations of ciprofloxacin exposure). Upon exposure to increasing ciprofloxacin concentrations, induction of the SOS response caused the cells to elongate significantly (from 3.1 $\pm$ 0.19µm on average in the absence of ciprofloxacin to 5.5 $\pm$ 0.71µm after 75 min of exposure to 30 ng/ml ciprofloxacin, Supp. Figure \ref{SIFig:Gam_cell_length}). Despite this, and due to compaction and centring, the nucleoid remained within $\sim$2 µm either side of the cell centre. The distribution of DNA-bound RecB molecules overlapped strongly with nucleoid density, indicating that as the nucleoid was compacted, newly formed DSBs triggered RecB recruitment at the centre of the cell, where the nucleoid was located. Furthermore, the close overlap of the distributions for DNA-bound RecB and nucleoid density strongly suggests that RecB binding occurs at random positions within the nucleoid, consistent with ciprofloxacin causing DSBs at random chromosomal locations.

\subsection*{Nucleoid compaction requires RecA loading}

The spatial distribution of nucleoid density and DNA-bound RecB was affected in the \dreca\ and \geneteneighty\ mutants (Figures \ref{Fig:nucleoid}C and \ref{Fig:nucleoid}D). In the \dreca\ mutant, exposure to ciprofloxacin did not trigger nucleoid compaction and centring as in wild-type cells. This confirmed that nucleoid compaction under ciprofloxacin exposure requires RecA loading, and most likely the induction of the SOS response. Despite the absence of compaction, the nucleoid occupied a smaller fraction of the cell in the \dreca\ mutant after exposure to ciprofloxacin (Supp. Figure \ref{SIFig:mutants_nucleoid_compaction}). This can be attributed to the progressive degradation of the bacterial chromosome by the repeated cycles of RecBCD binding (Figure \ref{Fig:recruitment}B).

In the \geneteneighty\ mutant, the nucleoid often formed irregular shapes and small segregated regions in the cell (Figure \ref{Fig:nucleoid}C). This disorganisation was especially visible in the presence of 30 ng/ml ciprofloxacin. Furthermore, in the \geneteneighty\ mutant the nucleoid did not undergo compaction within the 75 min of our experiment (Supp. Figure \ref{SIFig:mutants_nucleoid_compaction}). This might be a reflection of the less efficient RecA loading in this mutant, leading to different time dynamics of the SOS induction and either impaired or significantly delayed nucleoid compaction.

In both mutant strains, the localisation of DNA-bound RecB overlapped with the bacterial nucleoid (Figure \ref{Fig:nucleoid}D), indicating that RecB binding occurred at random positions within the nucleoid, similarly to the wild-type. In the \dreca\ mutant, addition of 30 ng/ml ciprofloxacin did not change the spatial distribution of nucleoid density or DNA-bound RecB. This result was expected, as \dreca\ cells were unable to induce the SOS response, and therefore did not undergo nucleoid compaction. In the \geneteneighty\ mutant in the absence of ciprofloxacin, both nucleoid density and DNA-bound RecB distributions were more centred than in wild-type cells, despite the nucleoid occupying the same fraction of the cell area (Supp. Figure \ref{SIFig:mutants_nucleoid_compaction}). This might be due to the slower growth rate of the \teneighty\ mutant resulting in a larger proportion of cells having a single centred nucleoid, rather than a bi-lobed nucleoid as seen in the wild-type and the \dreca\ mutant. The absence of change in the distribution of nucleoid density and DNA-bound RecB position in the \geneteneighty\ mutant upon exposure to ciprofloxacin reinforces the hypothesis that nucleoid compaction is impaired or delayed in this mutant.
