\section*{Introduction}

% Repairing DSBs is important for cells
% This is a very short paragraph, maybe even just 1-2 sentences to introduce what comes next
DNA double-strand breaks (DSBs) can arise from various sources (gamma radiation, antibiotic exposure, collision of the replication fork with DNA-bound proteins), and pose a serious threat to the integrity of bacterial cells. In \emph{E. coli}, DSBs are repaired through the RecBCD pathway

% DSBs can occur through different processes, and isolated instances are “easy” to deal with
% Review litt on endogenous damage? Michel & Leach papers
% Try to paint a picture of DSB repair that’s more complete than just “clean breaks” (i.e. include fork reversals)
% This could make for a good intro figure, e.g. 2 “simple cases”: fork reversal due to obstacle on the DNA, and clean break (because of UV, nuclease,…). This figure shouldn’t mention cipro/gyrase, because we don't really know what happens there.
% Emphasise that fork reversals are easier (and faster) to deal with, because of dsDNA digestion, or because the homologous sequence is very close


% Antibiotics such as ciprofloxacin can cause massive amounts of damage
% What is the mechanism of damage? Review cipro litt. (deletions that give sensitivity to cipro, known/unknown parts)
% How does the cell deal with that?
% We use different antibiotic concentrations to see the difference between low and high amounts of damage
% We look at our metrics over time to see how the damage is dealt with
% We use mutants to assess the importance of different parts of the repair mechanism


% In this paper, we aim to give more insight into specific steps of DSB repair
% Mechanism of ciprofloxacin DSB generation
% RecBCD binding time
% What happens when damage is too extensive to be repaired






